\documentclass[11pt]{article}

\usepackage{hyperref}
\usepackage{mathtools}
\usepackage{amsthm}
\usepackage{amssymb}
\usepackage{mathrsfs}
\usepackage[arrow]{xy}

\pagestyle{headings}

\theoremstyle{definition}
\newtheorem*{defn}{Definition}

\theoremstyle{definition}
\newtheorem{ex}{Example}

\theoremstyle{plain}
\newtheorem{theo}{Theorem}

\theoremstyle{plain}
\newtheorem{prop}{Proposition}

\theoremstyle{plain}
\newtheorem{lem}{Lemma}

\begin{document}

\author{Stephen Liu}
\title{2-Categories}
\date{August 15, 2018}

\maketitle

\begin{abstract}
Some introductory notes on 2-Categories.
\end{abstract}

The key thing to remember is that a 2-Category is just a category enriched in $\textbf{Cat}$, the category of small categories (A normal category is a category enriched in $\textbf{Set}$). So the hom-sets of a 2-Category are themselves a small category. First let's recast the definition of a category in terms of diagrams:

\begin{defn}[Category]
A (locally small) category $\mathscr{A}$ consists of a collection of

\begin{enumerate}
\item objects $A, B, C, \dots$
\item sets of morphisms $\mathscr{A}(A,B), \mathscr{A}(B,C), \dots$
\end{enumerate}

such that for every object $A$, there is a function, called \emph{identity}
\begin{equation*}
1_A: 1 \to \mathscr{A}(A,A)
\end{equation*}

and for every triple of objects $A, B, C$, there is a function, called \emph{composition}
\begin{equation*}
c_{ABC}: \mathscr{A}(A,B) \times \mathscr{A}(B,C) \to \mathscr{A}(A,C)
\end{equation*}

These functions obey the following commutative diagrams:

\begin{equation*}\tag*{\emph{Associativity Axiom}}
\begin{xy}
(0,0)*+{\mathscr{A}(A,B) \times \mathscr{A}(B,C) \times \mathscr{A}(C,D)}="v1";
(70,0)*+{\mathscr{A}(A,B) \times \mathscr{A}(B,D)}="v2";
(0,-30)*+{\mathscr{A}(A,C) \times \mathscr{A}(C,D)}="v3";
(70,-30)*+{\mathscr{A}(A,D)}="v4";
{\ar "v1";"v2"}?*!/_4mm/{1 \times c_{BCD}};
{\ar "v1";"v3"}?*!/^10mm/{c_{ABC} \times 1};
{\ar "v2";"v4"}?*!/_8mm/{c_{ABD}};
{\ar "v3";"v4"}?*!/^4mm/{c_{ACD}};
\end{xy}
\end{equation*}

\begin{equation*}\tag*{\emph{Identity Axiom}}
\begin{xy}
(0,0)*+{1 \times \mathscr{A}(A,B)}="v1";
(40,0)*+{\mathscr{A}(A,B)}="v2";
(80,0)*+{\mathscr{A}(A,B) \times 1}="v3";
(0,-30)*+{\mathscr{A}(A,A) \times \mathscr{A}(A,B)}="v4";
(40,-30)*+{\mathscr{A}(A,B)}="v5";
(80,-30)*+{\mathscr{A}(A,B) \times \mathscr{A}(B,B)}="v6";
{\ar "v2";"v1"}?*!/^4mm/{\cong};
{\ar "v2";"v3"}?*!/_4mm/{\cong};
{\ar "v1";"v4"}?*!/^8mm/{1_A \times 1};
{\ar "v3";"v6"}?*!/_8mm/{1 \times 1_B};
{\ar "v4";"v5"}?*!/^4mm/{c_{AAB}};
{\ar "v6";"v5"}?*!/_4mm/{c_{ABB}};
{\ar@{=} "v2";"v5"}
\end{xy}
\end{equation*}
\end{defn}

2-Categories have an analogous definition except that since we are enriched in \textbf{Cat}, $\mathscr{A}(A,B)$ are no longer sets, but rather small categories. Equivalently, $1_A, c_{ABC}$ are no longer functions but rather functors. With that, we have the following definition:

\begin{defn}[2-Category]
A 2-Category $\mathscr{A}$ consists of a collection of

\begin{enumerate}
\item objects, called 0-cells $A, B, C, \dots$
\item for each pair of 0-cells $A,B$ a small category $\mathscr{A}(A,B)$ where the objects of $\mathscr{A}(A,B)$ are our familiar morphisms $f:A \to B$, also called 1-cells, and the morphisms in $\mathscr{A}(A,B)$, $\alpha: f \Rightarrow g$ are called 2-cells.
\item for each 0-cell $A$, a functor
\begin{equation*}
1_A: 1 \to \mathscr{A}(A,A)
\end{equation*}
\item for each triple of 0-cells $A,B,C$ a bifunctor
\begin{equation*}
c_{ABC}: \mathscr{A}(A,B) \times \mathscr{A}(B,C) \to \mathscr{A}(A,C)
\end{equation*}
\end{enumerate}

such that the associativity and identity diagrams above commute.
\end{defn}

Since $\mathscr{A}(A,B)$ is a category and composition is a functor now, there's more stuff going on, which we need to take a look at:



\nocite{*}
\bibliographystyle{alpha}
\bibliography{ref2cats}

\end{document}