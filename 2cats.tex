\documentclass[11pt]{article}

\usepackage{hyperref}
\usepackage{mathtools}
\usepackage{amsthm}
\usepackage{amssymb}
\usepackage{mathrsfs}
\usepackage[arrow]{xy}

\pagestyle{headings}

\theoremstyle{definition}
\newtheorem*{defn}{Definition}

\theoremstyle{definition}
\newtheorem{ex}{Example}

\theoremstyle{plain}
\newtheorem{theo}{Theorem}

\theoremstyle{plain}
\newtheorem{prop}{Proposition}

\theoremstyle{plain}
\newtheorem{lem}{Lemma}

\begin{document}

\author{Stephen Liu}
\title{2-Categories}
\date{August 15, 2018}

\maketitle

\begin{abstract}
Some introductory notes on 2-Categories.
\end{abstract}

The key thing to remember is that a 2-Category is just a category enriched in $\textbf{Cat}$, the category of small categories (A normal category is a category enriched in $\textbf{Set}$). So the hom-sets of a 2-Category are themselves a small category. And that's all I have to say about that\dots

\nocite{*}
\bibliographystyle{alpha}
\bibliography{ref2cats}

\end{document}