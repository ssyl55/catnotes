\documentclass[11pt]{article}

\usepackage{hyperref}
\usepackage{mathtools}
\usepackage{amsthm}
\usepackage{mathrsfs}
\usepackage[arrow]{xy}

\pagestyle{headings}

\theoremstyle{definition}
\newtheorem*{defn}{Definition}

\theoremstyle{definition}
\newtheorem{ex}{Example}

\theoremstyle{plain}
\newtheorem{theo}{Theorem}

\theoremstyle{plain}
\newtheorem{prop}{Proposition}

\theoremstyle{plain}
\newtheorem{lem}{Lemma}

\begin{document}

\author{Stephen Liu}
\title{The Yoneda Lemma}
\date{June 18, 2018}

\maketitle

\begin{abstract}
Some notes on the Yoneda Lemma, starting with the notion of representable functors.
\end{abstract}

\subsection*{Covariant Representable Functors}

We first define a \emph{prototype} of a covariant representable functor out of a locally small category $\mathscr{A}$.

\begin{defn}
Let $\mathscr{A}$ be a locally small category, and fix $A \in \mathscr{A}$. Define the functor $H^{A}=\mathscr{A}(A, -): \mathscr{A} \to \textbf{Set}$ by the following mapping on

\begin{enumerate}
\item \textbf{objects}: For $B \in \mathscr{A}$, define $H^{A}(B) = \mathscr{A}(A,B)$, the hom-set of arrows in $\mathscr{A}$ from $A$ to $B$.
\item \textbf{morphisms}: For $g:B \to B'$, define $H^{A}(g) = \mathscr{A}(A, g): \mathscr{A}(A,B) \to \mathscr{A}(A,B')$ by $p \mapsto g \circ p$ for all $p : A \to B$, sending morphisms from $A$ to $B$ to morphisms from $A$ to $B'$ by post-composition with $g$.
\end{enumerate}
\end{defn}

From this prototype of a representable functor, we can now define covariant representable functors:

\begin{defn}[Covariant Representable Functor]
Let $\mathscr{A}$ be a locally small category. A functor $X:\mathscr{A} \to \textbf{Set}$ is representable if $X$ is naturally isomorphic to $H^{A}$ for some $A \in \mathscr{A}$. A representation of $X$ is a choice of an object $A$ along with a natural isomorphism from $H^{A}$ to $X$.
\end{defn}

Some examples of representable functors:

\begin{ex}[group $G$ regarded as a one object category]
Regarding a group $G$ as a one object category $\mathscr{G}$, with object $\ast$, $H^{\ast}=\ast(\ast, -): \mathscr{G} \to \textbf{Set}$ is a functor which maps
\begin{enumerate}
\item \textbf{objects}: There is only one object in $\mathscr{G}$, and $H^{\ast}(\ast)$ is the set of mappings from $\ast$ to $\ast$, otherwise known as the elements of the group $G$. So $H^{\ast}(\ast) \cong U(G)$ where $U:\textbf{Grp} \to \textbf{Set}$ is the forgetful functor that returns the underlying set of a group.
\item \textbf{morphisms}: Let $g \in G$, then $H^{\ast}(g)$ maps any element $h$ of $G$ to $g \circ h$, which, interpreted in the context of a group is just group mulitiplication on the left, i.e. $gh$.
\end{enumerate}

Since $\mathscr{G}$ has only one object, there is only one representable functor on it (up to isomorphism), and the representable is the underlying set $G$ acted on by left multiplication.
\end{ex}

\begin{ex}[$H^{1}:\textbf{Set} \to \textbf{Set}$]
\label{ex-set}
Let $1$ denote the set with just one element. The functor $1_{\textbf{Set}}$ is represented by the functor $H^{1}:\textbf{Set} \to \textbf{Set}$. To see this, let us first see what $H^{1}$ does and then show that it is naturally isomorphic to $1_{\textbf{Set}}$.

Let $B \in \textbf{Set}$, then $H^{1}(B) = \textbf{Set}(1, B) = \{  b:1 \to B \}$ the set of functions from the singleton set to $B$ that pick out an element of $B$. In particular, $H^{1}(B) \cong B$. Let $p: 1 \to B$ and $g: B \to B'$, then $H^{1}(g) = \textbf{Set}(1, g): \textbf{Set}(1, B) \to \textbf{Set}(1, B')$ sends functions $p$ that pick out an element of $B$ to functions $g \circ p$ that pick out an element of $B'$.

So we have the parallel functors $H^{1}, 1_{\textbf{Set}}: \textbf{Set} \to \textbf{Set}$ and we know what they do. Now we construct a natural isomorphism $t: H^{1} \Rightarrow 1_{\textbf{Set}}$. We already have that $H^{1}(B) \cong B$, so the components of $t$ are the isomorphisms $t_{B}: H^{1}(B) \to B$ (and of course $t^{-1}_{B}: B \to H^{1}(B))$. Now we need to check that the naturality square below commutes:

\begin{equation*}
\begin{xy}
(0, 0)*+{H^{1}(B)}="h1b"; (30, 0)*+{H^{1}(B')}="h1bp"; (0, -30)*+{B}="b"; (30, -30)*+{B'}="bp";
{\ar "h1b";"h1bp"}?*!/_3mm/{H^{1}(g)}; {\ar "b";"bp"}?*!/^3mm/{1_{\textbf{Set}}(g)=g};
{\ar "h1b";"b"}?*!/^3mm/{t_{B}}; {\ar "h1bp";"bp"}?*!/_3mm/{t_{B'}};
\end{xy}
\end{equation*}

where $g: B \to B'$.

It is enough to consider an element $b: 1 \to B$ from $H^{1}(B)$. $b$ picks out an element of $B$, which we will also identify by $b(1) = b$. Going down from $H^{1}(B)$, $t_{B}(b: 1 \to B)$ gives us this element $b$, which is then sent by $g$ to some element of $B'$, say $b'$. On the other hand, going right from $H^{1}(B)$, $H^{1}(g)(b: 1 \to B) = g \circ b$ which gives us the map $b': 1 \to B'$ that picks out $g(b(1)) = b'$. Finally, $t_{B'}(b')$ is precisely this element $b'$ of $B'$. So in fact the naturality square commutes.

Hence $1_{\textbf{Set}}$ is represented by $H^{1}$.
\end{ex}

\begin{ex}[$H^{\textbf{1}}:\textbf{Cat} \to \textbf{Set}$]
Similarly to the example above, $\text{ob}:\textbf{Cat} \to \textbf{Set}$ which sends a small category to its underlying set of objects is represented by $H^{\textbf{1}}$ where $\textbf{1}$ is the one-object category. This is because when $\mathscr{B}$ is a category, $H^{\textbf{1}}(\mathscr{B}) = \textbf{Cat}(\textbf{1}, \mathscr{B})$ which picks out objects of $\mathscr{B}$ and is isomorphic to $\text{ob}\mathscr{B}$. The proof that these functors are naturally isomorphic is essentially the same as the one for $1_{\textbf{Set}}$ and $H^{1}$ in the example above.
\end{ex}

\subsubsection*{Adjoints and Representables}

Now we establish the claim that any set valued functor with a left adjoint is representable. We first prove the following lemma.

\begin{lem}
Let $\mathscr{A}, \mathscr{B}$ be locally small categories with functors $F:\mathscr{A} \to \mathscr{B}, G:\mathscr{B} \to \mathscr{A}$ such that $F \dashv G$. Fix $A \in \mathscr{A}$, then the functor \begin{equation*} \mathscr{A}(A, G(-)): \mathscr{B} \to \textbf{Set} \end{equation*} that is, the composition $\mathscr{B} \overset{G}{\to} \mathscr{A} \overset{H^{A}}{\to} \textbf{Set}$ is representable.
\end{lem}

\begin{proof}
Because $F \dashv G$, we have the isomorphism $t^{-1}_{A,B}: \mathscr{A}(A, G(B)) \to \mathscr{B}(F(A), B)$ for each $B \in \mathscr{B}$. Now $H^{F(A)}$ maps $B$ to $\mathscr{B}(F(A),B)$, and it maps a morphism $g: B \to B'$ to $\mathscr{B}(F(A),g):\mathscr{B}(F(A),B) \to \mathscr{B}(F(A),B')$. So we suspect that the isomorphisms $t^{-1}_{A,B}$ gives rise to the natural isomorphism $t^{-1}:\mathscr{A}(A,G(-)) \Rightarrow H^{F(A)}$. We show that this is in fact the case.

We need to check naturality. Let $g:B \to B'$. We need to check that the following naturality square commutes:

\begin{equation*}
\begin{xy}
(0,0)*+{\mathscr{A}(A, G(B))}="v1"; (60,0)*+{\mathscr{A}(A, G(B'))}="v2"; (0,-30)*+{\mathscr{B}(F(A), B)}="v3"; (60,-30)*+{\mathscr{B}(F(A), B')}="v4";
{\ar "v1";"v2"}?*!/_5mm/{\mathscr{A}(A, Gg)=H^{A}(Gg)};
{\ar "v1";"v3"}?*!/^5mm/{t^{-1}_{A,B}};
{\ar "v2";"v4"}?*!/_5mm/{t^{-1}_{A,B'}};
{\ar "v3";"v4"}?*!/^5mm/{H^{F(A)}(g)};
\end{xy}
\end{equation*}

It suffices to consider a map $f: A \to G(B)$ from $\mathscr{A}(A,G(B))$ as it travels around the diagram. Going down from $\mathscr{A}(A,G(B))$, $t^{-1}_{A,B}(f:A \to G(B))$ is a map $t^{-1}_{A,B}(f):F(A) \to B$, and then $H^{F(A)}(g)(t^{-1}_{A,B}) = g \circ t^{-1}_{A,B}$ which is a map from $F(A)$ to $B'$. On the other hand, going right from $\mathscr{A}(A,G(B))$, $H^{A}(Gg)(f:A \to G(B)) = Gg \circ f$ which is a map from $A$ to $G(B')$ and then $t^{-1}_{A,B'}(Gg \circ f)$ is a map from $F(A)$ to $B'$. So the question is whether $t^{-1}_{A,B'}(Gg \circ f) = g \circ t^{-1}_{A,B}(f)$, but that is precisely what it means for $t^{-1}$ to be natural with respect to $B$ (See \cite{liu_adjoints_2018}), so in fact $\mathscr{A}(A,G(-)) \cong H^{F(A)}$ and is therefore representable.
\end{proof}

Now we are ready to prove the main claim of this subsection:

\begin{theo}
Any set-valued functor that has a left adjoint is representable.
\end{theo}

\begin{proof}
Let $G:\mathscr{A}\to\textbf{Set}$ be a functor with left adjoint $F$. Let $1$ denote the set with a single element. For all $A \in \mathscr{A}$, $G(A)$ is a set and by example \ref{ex-set} above, $H^{1}(G(A)) = \textbf{Set}(1, G(A)) \cong G(A)$ naturally in $A$ where $H^{1}$ is our functor from $\textbf{Set}$ to $\textbf{Set}$ in example \ref{ex-set} above. So $G \cong \textbf{Set}(1, G(-))$ and by the lemma above, this means $G$ is representable.
\end{proof}

\subsubsection*{Covariant Embedding}

\begin{prop}
Let $f:A' \to A$, then $f$ induces a natural transformation $H^{f}: H^{A} \Rightarrow H^{A'}$ with components $H^{f}_{B}:H^{A}(B) \to H^{A'}(B)$ so that a map $p:A \to B$ in $H^{A}(B)$ gets mapped via precomposition with $f$ to $p \circ f: A' \to B$.
\end{prop}

\begin{proof}
We need to show that this map $H^{f}$ is indeed a natural transformation. Let $g:B\to B'$. This means checking that the naturality square:

\begin{equation*}
\begin{xy}
(0,0)*+{H^{A}(B)}="v1";(30,0)*+{H^{A}(B')}="v2";(0,-30)*+{H^{A'}(B)}="v3";(30,-30)*+{H^{A'}(B')}="v4";
{\ar "v1";"v2"}?*!/_4mm/{H^{A}(g)};
{\ar "v1";"v3"}?*!/^4mm/{H^{f}_{B}};
{\ar "v2";"v4"}?*!/_4mm/{H^{f}_{B'}};
{\ar "v3";"v4"}?*!/^4mm/{H^{A'}(g)};
\end{xy}
\end{equation*}

commutes.

It is sufficient to consider a map $p:A \to B$ in $H^{A}(B)$ as it travels around the square. Going down, we have first $H^{f}_{B}(p) = p \circ f$, and then to the right we have $H^{A'}(g)(p \circ f) = g \circ (p \circ f)$. Going to the right we have $H^{A}(g)(p) = g \circ p$, and then going down we have $H^{f}_{B'}(g \circ p) = (g \circ p) \circ f$. However, since morphism composition is associative, these two things are equal and so the square indeed commutes.
\end{proof}

With this natural transformation between covariant representables, we have the following definition:

\begin{defn}
The \emph{covariant embedding} is the functor $H^{\bullet}:\mathscr{A}^{\text{op}} \to [\mathscr{A},\textbf{Set}]$ defined by the following mapping on:
\begin{enumerate}
\item \textbf{objects: } For object $A$ in $\mathscr{A}$, $H^{\bullet}(A) = H^{A}$.
\item \textbf{morphisms: } For morphism $f:A' \to A$, $H^{\bullet}(f) = H^{f}$.
\end{enumerate}
\end{defn}

We can do many of the same things with the dual case.

\subsection*{Contravariant Representable Functors}

Again, we start with a definition of the \emph{prototypical} contravariant representable functor:

\begin{defn}
Let $\mathscr{A}$ be a locally small category with object $A$. Define the functor $H_{A}:\mathscr{A}(-,A):\mathscr{A}^{\text{op}} \to \textbf{Set}$ by the following mapping on
\begin{enumerate}
\item \textbf{objects:} For object $B$ in $\mathscr{A}$, define $H_{A}(B)=\mathscr{A}(B,A)$.
\item \textbf{morphisms:} For morphism $g:B' \to B$, define $H_{A}(g)=\mathscr{A}(g,A):\mathscr{A}(B,A) \to \mathscr{A}(B',A)$ by $p:B \to A \mapsto p \circ g: B' \to B \to A$.
\end{enumerate}
\end{defn}

Similar to the covariant case, any functor $F:\mathscr{A}^{\text{op}} \to \textbf{Set}$ that is naturally isomorphic to a \emph{prototypical} contravariant representable functor is called contravariantly representable.

\begin{ex}
The functor $\mathscr{P}:\textbf{Set}^{\text{op}} \to \textbf{Set}$ which maps sets to their powersets and maps functions $g:B' \to B$ to $\mathscr{P}(g)=g^{-1}U$ for all $U \in \mathscr{P}(B)$ where $g^{-1}U=\{x' \in B' | g(x') \in U\}$ is naturally isomorphic to $H_{2}:\textbf{Set}^{\text{op}} \to \textbf{Set}$ where $2$ denotes the set with two elements.

To show this, we first need to realize that a subset of a set $B$ is just a boolean mapping from the set to $2$ ($f(b)=1$ means $b$ is in that subset). So $H_2(B)=\textbf{Set}(B,2) \cong \mathscr{P}(B)$. Define $t_{B}:H_2(B) \to \mathscr{P}(B)$ with this isomorphism. We show that the $t_B$'s are natural in $B$ and therefore define the components of a natural transformation $t:H_2 \Rightarrow \mathscr{P}$. Let $B,B'$ be sets, and let $g$ be a map from $B'$ to $B$. To do this, we show that the following naturality square commutes:

\begin{equation*}
\begin{xy}
(0,0)*+{H_2(B)}="h2b"; (30,0)*+{H_2(B')}="h2bp"; (0,-30)*+{\mathscr{P}(B)}="pb"; (30,-30)*+{\mathscr{P}(B')}="pbp";
{\ar "h2b";"pb"}?*!/^4mm/{t_B};
{\ar "h2b";"h2bp"}?*!/_4mm/{H_2(g)};
{\ar "h2bp";"pbp"}?*!/_4mm/{t_{B'}};
{\ar "pb";"pbp"}?*!/^4mm/{\mathscr{P}(g)};
\end{xy}
\end{equation*}

Consider a map $u:B \to 2$ that picks out an associated subset $U \subseteq B$. $t_B(u)$ gives precisely this subset $U$, and $\mathscr{P}(g)(U)$ gives the subset $U'$ of $B'$ that maps along $g$ into $U$. On the other hand, $H_2(g)(u)$ precomposes $g$ with $u$ to give the mapping $u \circ g: B' \to B \to 2$ which picks out the subset $U'$ of $B'$ that, when mapped along $g$ gives back $U$. And $t_B'(u \circ g)$ gives this $U'$. So the diagram commutes.

Hence, $\mathscr{P}$ is represented by $H_2$.
\end{ex}

\subsubsection*{Contravariant Embedding}

Similar to the covariant case, the map $f:A \to A'$ induces a natural transformation $H_f: H_A \Rightarrow H_{A'}$ with components $H_f^B:H_A(B) \to H_{A'}(B)$ such that a map $p:B \to A \mapsto f \circ p: B \to A \to A'$.

This gives rise to the following contravariant embedding (which we will call the Yoneda Embedding)

\begin{defn}[Yoneda Embedding]
Let $\mathscr{A}$ be a locally small category. The Yoneda Embedding is the functor $H_{\bullet}:\mathscr{A} \to [\mathscr{A}^{\text{op}},\textbf{Set}]$ defined by the following mapping on:

\begin{enumerate}
\item \textbf{objects: } For object $A$ in $\mathscr{A}$, $H_{\bullet}(A) = H_A$.
\item \textbf{morphisms: } For morphism $f: A \to A'$, $H_{\bullet}(f) = H_f$.
\end{enumerate}
\end{defn}

\begin{prop}
$H_{\bullet}$ is injective on isomorphism classes of objects.
\end{prop}

\begin{proof}
Let $A,A'$ be objects in $\mathscr{A}$ such that $H_A \cong H_{A'}$. We need to show $A \cong A'$, that is, find maps $f: A \to A'$, $g: A' \to A$ that are mutually inverse. Since $H_A \cong H_{A'}$ that means there are maps (natural transformations) $\alpha:H_A \Rightarrow H_{A'}$ and $\beta:H_{A'} \Rightarrow H_A$ such that $\alpha \circ \beta = 1_{H_{A'}}$ and $\beta \circ \alpha = 1_{H_A}$. We need to use $\alpha$ and $\beta$ to construct the $f$ and $g$ we need. 

First let's consider the components of $\alpha$. Let $B$ be an object in $\mathscr{A}$. Then $\alpha_B$ is a map from $H_A(B) \to H_{A'}(B)$. Letting $B=A$, we have $\alpha_A:H_A(A) \to H_{A'}(A)$. Now, we don't know what morphisms there are in $H_A(A)$, but we certainly know that $1_A$ exists. And applying $\alpha_A$ to $1_A$, we get a map $\alpha_A(1_A):A \to A'$.

Similarly, considering $\beta_{A'}$ applied to $(1_{A'})$, we get a map $\beta_{A'}(1_{A'}):A' \to A$.

So now we have our maps between $A$ and $A'$. We need to show that they are mutually inverse.

Naturality of $\alpha$ in $B$ means that for every $p: B \to B'$ the following naturality square holds (note the contravariance):

\begin{equation*}
\begin{xy}
(0,0)*+{H_A(B)}="v1"; (30,0)*+{H_A(B')}="v2"; (0,-30)*+{H_{A'}(B)}="v3"; (30,-30)*+{H_{A'}(B')}="v4";
{\ar "v1";"v3"}?*!/^4mm/{\alpha_B};
{\ar "v2";"v1"}?*!/^4mm/{- \circ p};
{\ar "v2";"v4"}?*!/_4mm/{\alpha_{B'}};
{\ar "v4";"v3"}?*!/_4mm/{- \circ p};
\end{xy}
\end{equation*}

In particular, if $x:B' \to A$ is a map $H_A(B')$, then $\alpha_B(x \circ p) = \alpha_{B'}(x) \circ p$.

Again, what we need is a specialization of this general phenomenon. Letting $x=1_A$, which means making $B = A'$ and $B' = A$, our $p$ becomes a map from $A' \to A$ and by naturality above we have $\alpha_{A'}(1_A \circ p) = \alpha_A(1_A) \circ p$. Now $p$ is a map from $A' \to A$, so letting $p = \beta_{A'}(1_{A'})$ we have that $\alpha_A(1_A) \circ \beta_{A'}(1_{A'}) = \alpha_{A'}(1_A \circ \beta_{A'}(1_{A'})) = \alpha_{A'}(\beta_{A'}(1_{A'}))$. But since $\alpha_{A'}$ and $\beta_{A'}$ are mutually inverse, this just means $\alpha_A(1_A) \circ \beta_{A'}(1_{A'}) = 1_{A'}$, which is precisely what we wanted.

Similarly, we can show using the naturality of $\beta$ and specializing to $1_{A'}$ that $\beta_{A'}(1_{A'}) \circ \alpha_A(1_A) = 1_A$.

Thus $A \cong A'$.
\end{proof}

\subsection*{The Yoneda Lemma}

\begin{theo}[The Yoneda Lemma]
Let $\mathscr{A}$ be a locally small category. Then $[\mathscr{A}^{\text{op}}, \textbf{Set}](H_A,X) \cong X(A)$ naturally in $A \in \mathscr{A}$ and $X \in [\mathscr{A}^{\text{op}}, \textbf{Set}]$.
\end{theo}

\begin{proof}
First let's give a quick outline of the proof:
\begin{enumerate}
\item Produce a mapping $t_{A,X}: [\mathscr{A}^{\text{op}}, \textbf{Set}](H_A, X) \to XA$.
\item Produce a mapping $t^{-1}_{A,X}: XA \to [\mathscr{A}^{\text{op}}, \textbf{Set}](H_A,X)$.
\item Show that $t,t^{-1}$ are mutually inverse.
\item Show that $t$ is natural in $A$.
\item Show that $t$ is natural in $X$.
\end{enumerate}

We're not going to do these steps in order, instead we'll do these steps the way I did it when I came up with the proof, so you'll see my motivation for each successive step.

Let's begin with step 1. Let $\alpha:H_A \Rightarrow X$ be a natural transformation. Looking at $\alpha$'s components, $\alpha_B:H_A(B) \to XB$ sends a map $g:B \to A$ to the element of $XB$, $\alpha_B(g)$. $t_{A,X}$ needs to map $\alpha$ to one specific element of $XA$. Consider the special case where $B=A$. Then we have the component $\alpha_A:H_A(A) \to XA$. Since the only map we are certain exists in $H_A(A)$ is the identity map $1_A$, this suggests we need to define $t_{A,X}(\alpha) = \alpha_A(1_A)$.

To check that this mapping is the correct one, let's first check that $t_{A,X}$ is natural in $X$. Let $F:X \Rightarrow X'$ be a natural transformation. $t_{A,X}$ natural in $X$ means that the following naturality square holds:

\begin{equation*}
\begin{xy}
(0,0)*+{[\mathscr{A}^{\text{op}},\textbf{Set}](H_A,X)}="v1"; (60,0)*+{[\mathscr{A}^{\text{op}},\textbf{Set}](H_A,X')}="v2"; (0,-30)*+{XA}="v3"; (60,-30)*+{X'A}="v4";
{\ar "v1";"v2"}?*!/_4mm/{F \circ -};
{\ar "v1";"v3"}?*!/^6mm/{t_{A,X}};
{\ar "v2";"v4"}?*!/_6mm/{t_{A,X'}};
{\ar "v3";"v4"}?*!/^4mm/{F_A};
\end{xy}
\end{equation*}

Let $\alpha:H_A \Rightarrow X$ be a natural transformation. Then going down first and then to the right via $F_A$ we have $F_A(t_{A,X}(\alpha)) = F_A(\alpha_A(1_A))$. On the other hand going right first via $F \circ -$ and then down we have $t_{A,X'}(F \circ \alpha) = (F \circ \alpha)_A(1_A)$. And by compositionality, $(F \circ \alpha)_A(1_A) = F_A \circ \alpha_A(1_A) = F_A(\alpha_A(1_A))$. So $t_{A,X}$ is indeed natural in $X$.

Now let us check that $t_{A,X}$ is natural in $A$. We won't succeed yet but this will give us a hint as to how we should define $t_{A,X}^{-1}$. Let $f$ be a map from $A$ to $A'$. We need to check that the following naturality square commutes:

\begin{equation*}
\begin{xy}
(0,0)*+{[\mathscr{A}^{\text{op}},\textbf{Set}](H_A,X)}="v1"; (60,0)*+{[\mathscr{A}^{\text{op}},\textbf{Set}](H_{A'},X)}="v2"; (0,-30)*+{XA}="v3"; (60,-30)*+{XA'}="v4";
{\ar "v2";"v1"}?*!/^4mm/{- \circ H_f};
{\ar "v1";"v3"}?*!/^6mm/{t_{A,X}};
{\ar "v2";"v4"}?*!/_6mm/{t_{A',X}};
{\ar "v4";"v3"}?*!/_4mm/{Xf};
\end{xy}
\end{equation*}

Let $\alpha':H_{A'} \Rightarrow X$ be a natural transformation. Then going down first and then left via $Xf$ we have $Xf(t_{A',X}(\alpha')) = Xf(\alpha'_{A'}(1_{A'}))$. On the other hand, going to the left first via $- \circ H_f$ and then down, we have $t_{A,X}(\alpha' \circ H_f) = (\alpha'_{A} \circ H_f^A)(1_A)=\alpha'_{A}(f)$ (Because $H_f^A(1_A) = f \circ 1_A) = f$. So we need to have the equality

\begin{equation*}
Xf(\alpha'_{A'}(1_{A'})) = \alpha'_{A}(f)
\end{equation*}

Which we can't prove just yet. However, this does give us a hint as to how we should define $t_{A,X}^{-1}$. Notice that $\alpha'_{A'}(1_A')$ is an element of $XA'$ and that for every $g:B \to A' \in H_{A'}(B)$, $g \mapsto Xg(\alpha'_{A'}(1_A'))$ defines a mapping from $H_{A'}(B)$ to $XB$, which is beginning to look like the components of a natural transformation $H_{A'} \Rightarrow X$.

With that let us define our mapping $t_{A,X}^{-1}:XA \to [\mathscr{A}^{\text{op}}, \textbf{Set}](H_A,X)$. Let $x$ be an element of $XA$. We define $t_{A,X}^{-1}(x) = \left( X(f: B \to A)(x) \right)_{B \in \mathscr{A}}$. So for each $B$, we have a family of morphisms $H_A(B) \to XB$ given by $f \mapsto Xf(x)$. Now we need to check that this family of morphisms does indeed form a natural transformation $H_A \Rightarrow X$. 

Let $k: B \to B'$. We need to check that the following naturality square commutes:

\begin{equation*}
\begin{xy}
(0,0)*+{H_A(B)}="v1"; (30,0)*+{H_A(B')}="v2"; (0,-30)*+{XB}="v3"; (30,-30)*+{XB'}="v4";
{\ar "v2";"v1"}?*!/^4mm/{- \circ k};
{\ar "v1";"v3"}?*!/^18mm/{X(-:B \to A)(x)};
{\ar "v2";"v4"}?*!/_18mm/{X(-':B' \to A)(x)};
{\ar "v4";"v3"}?*!/_4mm/{Xk};
\end{xy}
\end{equation*}

Let $p: B' \to A$. Then going left and down we first have $p \circ k: B \to A$ which then becomes $X(p \circ k)(x)$. On the other hand, going down first and then left we have $Xk(Xp(x))$. These expressions are equal by the (co)functoriality of $X$. So $t_{A,X}^{-1}(x)$ does indeed yield a natural transformation.

Now we need to check that $t_{A,X}$ and $t_{A,X}^{-1}$ are in fact mutually inverse. Let $x \in XA$. Then $t_{A,X}(t_{A,X}^{-1}(x)) = t_{A,X}(X(-)(x)) = (X(-)(x))_A(1_A) = X1_A(x) = 1_{XA}(x)=x$. So $t_{A,X} \circ t_{A,X}^{-1}$ does indeed equal $1_{XA}$. The other direction is a little bit harder:

Let $\alpha$ be a natural transformation from $H_A \Rightarrow X$. $t_{A,X}^{-1}(t_{A,X}(\alpha)) = t_{A,X}^{-1}(\alpha_A(1_A)) = X(-)(\alpha_A(1_A))$. Let $f: B \to A$. For $X(-)(\alpha_A(1_A)) = \alpha$, we need to have $Xf(\alpha_A(1_A)) = \alpha_B(f)$ for every $f$. The only tool we have at our disposal is naturality, either naturality of $X(-)(x)$ which we just established, or the naturality of $\alpha$. When I first went about trying to prove this I thought we ought to use the naturality of $X(-)(x)$, however it turns out that the correct thing to do is use the naturality of $\alpha$. (Thank you \cite{leinster_basic_2014}!) Let's write out the naturality square for $\alpha$ with $f$:

\begin{equation*}
\begin{xy}
(0,0)*+{H_A(B)}="v1"; (30,0)*+{H_A(A)}="v2"; (0,-30)*+{XB}="v3"; (30,-30)*+{XA}="v4";
{\ar "v2";"v1"}?*!/^4mm/{- \circ f};
{\ar "v1";"v3"}?*!/^4mm/{\alpha_B};
{\ar "v2";"v4"}?*!/_4mm/{\alpha_A};
{\ar "v4";"v3"}?*!/_4mm/{Xf};
\end{xy}
\end{equation*}

Considering the image of $1_A$ as it moves around the square, we have $\alpha_B(1_A \circ f) = \alpha_B(f) = Xf(\alpha_A(1_A))$, which is exactly what we needed.

So $t$ and $t^{-1}$ are mutually inverse, and in fact this last step is also what we needed to prove that $t_{A,X}$ is natural in $A$, so we are done.
\end{proof}

\subsection*{Consequences of the Yoneda Lemma}

One consequence of the Yoneda Lemma is that it justifies the name of the Yoneda Embedding, in that the Yoneda Embedding embeds $\mathscr{A}$ into a full subcategory of its presheaf category:

\begin{prop}
The Yoneda Embedding: $H_{\bullet}: \mathscr{A} \to [\mathscr{A}^{\text{op}}, \textbf{Set}]$ is full and faithful.
\end{prop}

\begin{proof}
$H_{\bullet}$ being full and faithful is another way of saying the map $K: \mathscr{A}(A,A') \to [\mathscr{A}^{\text{op}}, \textbf{Set}](H_{\bullet}(A), H_{\bullet}(A'))$ is bijective for all $A,A' \in \mathscr{A}$. First recall that $H_{\bullet}(A) = H_A$, $H_{\bullet}(A') = H_{A'}$ and that $\mathscr{A}(A,A') = H_{A'}(A)$. So $K$ is actually a map from $H_{A'}(A) \to [\mathscr{A}^{\text{op}}, \textbf{Set}](H_A, H_{A'})$. Remember that $K$ comes from the Yoneda Embedding, which means $K$ takes a morphism $f: A \to A'$ and maps it to $H_f: H_A \to H_{A'}$.

Well, by the Yoneda Lemma, we know there is a bijection $t_{A,H_{A'}}^{-1}: H_{A'}(A) \to [\mathscr{A}^{\text{op}}, \textbf{Set}](H_A, H_{A'})$. If $K$ and $t_{A,H_{A'}}^{-1}$ were actually the same, then that would show that $K$ is bijective. Let's consider what $t_{A, H_{A'}}^{-1}$ actually does. Let $f: A \to A'$. Then $f \in H_{A'}(A)$ and $t_{A,H_{A'}}^{-1}(f) = H_{A'}(g)(f) = f \circ g$ for all $g: B \to A$. So $t_{A,H_{A'}}^{-1}(f)$ is post-composition with $f$, which is exactly what $H_f$ does, so $K$ and $t_{A,H_{A'}}^{-1}$ are in fact the same.
\end{proof}

\nocite{*}
\bibliographystyle{alpha}
\bibliography{refyoneda}

\end{document}