\documentclass[11pt]{article}

\usepackage{hyperref}
\usepackage{mathtools}
\usepackage{amsthm}
\usepackage{mathrsfs}
\usepackage[arrow]{xy}

\pagestyle{headings}

\theoremstyle{definition}
\newtheorem*{defn}{Definition}

\theoremstyle{definition}
\newtheorem{ex}{Example}

\theoremstyle{plain}
\newtheorem{theo}{Theorem}

\theoremstyle{plain}
\newtheorem{prop}{Proposition}

\theoremstyle{plain}
\newtheorem*{lem}{Lemma}

\begin{document}

\author{Stephen Liu}
\title{The Yoneda Lemma}
\date{June 18, 2018}

\maketitle

\begin{abstract}
Some notes on the Yoneda Lemma, starting with the notion of representable functors.
\end{abstract}

We first define a \emph{prototype} of a representable functor out of a locally small category $\mathscr{A}$.

\begin{defn}
Let $\mathscr{A}$ be a locally small category, and fix $A \in \mathscr{A}$. Define the functor $H^{A}=\mathscr{A}(A, -): \mathscr{A} \to \textbf{Set}$ by the following mapping on

\begin{enumerate}
\item \textbf{objects}: For $B \in \mathscr{A}$, define $H^{A}(B) = \mathscr{A}(A,B)$, the hom-set of arrows in $\mathscr{A}$ from $A$ to $B$.
\item \textbf{morphisms}: For $g:B \to B'$, define $H^{A}(g) = \mathscr{A}(A, g): \mathscr{A}(A,B) \to \mathscr{A}(A,B')$ by $p \mapsto g \circ p$ for all $p : A \to B$, sending morphisms from $A$ to $B$ to morphisms from $A$ to $B'$ by post-composition with $g$.
\end{enumerate}
\end{defn}

This is the covariant version of $H$, there is a contravariant version, denoted $H_{A}$ which does pre-composition instead.

From this prototype of a representable functor, we can now define representable functors:

\begin{defn}[Representable Functor]
Let $\mathscr{A}$ be a locally small category. A functor $X:\mathscr{A} \to \textbf{Set}$ is representable if $X$ is naturally isomorphic to $H^{A}$ for some $A \in \mathscr{A}$. A representation of $X$ is a choice of an object $A$ along with a natural isomorphism from $H^{A}$ to $X$.
\end{defn}

Some examples of representable functors:

\begin{ex}[group $G$ regarded as a one object category]
\end{ex}

\begin{ex}[$H^{1}:\textbf{Set} \to \textbf{Set}$]
\end{ex}

\begin{ex}[$H^{\textbf{1}}:\textbf{Cat} \to \textbf{Set}$]
\end{ex}

\nocite{*}
\bibliographystyle{alpha}
\bibliography{refyoneda}

\end{document}