\documentclass[11pt]{article}

\usepackage{mathtools}
\usepackage{amsthm}
\usepackage{amssymb}
\usepackage{mathrsfs}
\usepackage[arrow]{xy}

\pagestyle{headings}

\theoremstyle{definition}
\newtheorem*{defn}{Definition}

\theoremstyle{plain}
\newtheorem{theo}{Theorem}

\theoremstyle{plain}
\newtheorem{prop}{Proposition}

\theoremstyle{plain}
\newtheorem*{lem}{Lemma}

\begin{document}

\author{Stephen Liu}
\title{Abelianization of Groups}
\date{June 11, 2018}

\maketitle

\begin{defn}[Commutator]
Let $G$ be a group and $x,y \in G$. Then the commutator of $x$ and $y$, is $[x,y] = x^{-1}y^{-1}xy$.
\end{defn}

It is easy to see from this definition that $xy=yxx^{-1}y^{-1}xy=yx[x,y]$ and that $xy=yx$ if and only if $[x,y]=1$.

\begin{defn}[Commutator Subgroup]
Suppose $A,B$ are nonempty subsets of $G$, then $[A,B]=\langle[a,b]| a \in A, b \in B\rangle$ is the group generated by commutators of elements from $A$ and from $B$. In particular, $G'=[G,G]=\langle[x,y]|x,y \in G\rangle$ is called the commutator subgroup of $G$.
\end{defn}

This commutator subgroup has nice properties which allow us to "abelianize" $G$, that is, get an abelian group from $G$. Since we saw above that $x,y$ commute if and only if $[x,y]=1$, this suggests that to get an abelian group from $G$, we need some way of \emph{setting} $[x,y]$ to $1$, which suggests taking the quotient by $G'$. And indeed we have the following proposition:

\begin{prop}
$G/G'$ is abelian.
\end{prop}

\begin{proof}
Let $xG',yG' \in G/G'$, since $xy=yx[x,y]$ and $[x,y] \in G'$, we have $(xG')(yG')=(xyG')=(yx[x,y]G')=(yxG')=(yG')(xG')$.
\end{proof}

So we were able to "abelianize" $G$. However, $G'$ is also nice in that it has the property that it is the smallest normal subgroup of $G$ such that $G/G'$ is abelian. In other words, $G/G'$ is the largest abelian quotient of $G$. Precisely, we have the following proposition:

\begin{prop}
$H \trianglelefteq G$ and $G/H$ abelian if and only if $G' \leq H$.
\end{prop}

\begin{proof}
First suppose $H \trianglelefteq G$ and $G/H$ is abelian. 

So $1H=(xH)^{-1}(yH)^{-1}(xH)(yH)=(x^{-1}y^{-1}xyH)=([x,y]H)$ which means $[x,y] \in H$ for all $x,y \in G$, so $G' \leq H$.

Now suppose $G' \leq H$, then since $G/G'$ is abelian, every subgroup is normal and so $G/H \trianglelefteq G/G'$ and by the lattice isomorphism theorem that means $H \trianglelefteq G$. Also, by the third isomorphism theorem, we have $G/H \cong (G/G')/(H/G')$ which means $G/H$ is abelian since it is the quotient of an abelian group ($G/G'$).
\end{proof}

Additionally we have this nice universal property regarding $G/G'$ which is that for any abelian group $A$ and group homomorphism $\phi : G \to A$, $G' \subseteq \text{Ker}\phi$ and the following diagram commutes:

\begin{equation*}
\begin{xy}
(0,0)*+{G}="G"; (30,0)*+{G/G'}="gprime"; (30,-30)*+{A}="A";
{\ar "G";"gprime"}?*!/_3mm/{\eta};
{\ar "G";"A"}?*!/^3mm/{\phi};
{\ar@{-->} "gprime";"A"}?*!/_6mm/{\exists ! \overline{\phi}}
\end{xy}
\end{equation*}

\begin{proof}
It is easy enough to show $G' \subseteq \text{Ker}\phi$ 

since $\phi(x^{-1}y^{-1}xy)=\phi(x^{-1})\phi(y^{-1})\phi(x)\phi(y)=\phi(x)^{-1}\phi(y)^{-1}\phi(x)\phi(y)$ and since $A$ is abelian, this equals $1_{A}$. We define $\overline{\phi} : G/G' \to A$ by $\overline{\phi}(xG')=\phi(x)$. Now we need to show $\overline{\phi}$ exists and is unique and commutes according to the diagram above. To show that it exists, we show that it is a well defined group homomorphism. Suppose $xG'=yG'$, for $\overline{\phi}$ to be well defined, we need $\overline{\phi}(xG')=\overline{\phi}(yG')$. Since $xG'=yG'$, $xy^{-1} \in G' \subseteq \text{Ker}\phi$, so $\phi(x)\phi(y)^{-1}=\phi(x)\phi(y^{-1})=\phi(xy^{-1})=1_{A}$, so $\phi(x)=\phi(y)$, which by the definition of $\overline{\phi}$, means $\overline{\phi}(xG')=\overline{\phi}(yG')$. $\overline{\phi}$ is indeed a group homomorphism because $\overline{\phi}(xG'yG')=\overline{\phi}(xyG')=\phi(xy)=\phi(x)\phi(y)=\overline{\phi}(xG')\overline{\phi}(yG')$ and $\overline{\phi}$ is unique because it is completely determined and defined by $\phi$. Finally, $\phi = \overline{\phi} \circ \eta$ because $\overline{\phi}(\eta(x))=\overline{\phi}(xG')=\phi(x)$.
\end{proof}

\end{document}